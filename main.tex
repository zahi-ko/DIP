\documentclass[bachelor]{thesis-uestc}
\usepackage{verbatim}
\usepackage{amssymb}
\usepackage{minted}
\usepackage{booktabs}

% \setCJKmonofont{fzhei.ttf}
\title{数字图像处理课程设计报告}{Course Design Report on Digital Image Processing}

\author{pt}{1}
\advisor{蒲恬\chinesespace 高级工程师}{Tian Pu, Senior Engineer}
\school{信息与通信工程学院}{School of Information and Communication Engineering}
\major{通信工程}{Communication Engineering}
\studentnumber{131313}

\begin{document}

\makecover

\begin{chineseabstract}
本课程设计以“基于 Weber 对比度的图像增强方法”为主题,旨在综合运用在《数字图像处理》课程中学习到的图像处理理论与编程实践能力,完成一个完整的应用案例。设计过程中,首先对 Weber 对比度的计算与可视化方法进行了研究与探讨,包括采用不同滤波器获取背景、通过直方图、热力图等多种可视化方法展示计算获得的 Weber 对比度等;随后基于计算得到的 Weber 对比度,尝试进行图像增强,获得的结果符合预期目标,验证了设计算法的有效性;最后尝试将算法扩展到 Michelson 对比度和均方根平均值对比度,但与 Weber 对比度相比,图像的增强效果较差。通过本次课程设计,进一步加深了对图像处理理论的理解,提升了编程实现与系统调试的综合能力,为后续更加复杂的实际应用与研究奠定了基础。


\chinesekeyword{Weber 对比度,图像增强,Michelson 对比度,均方根平均值对比度,图像滤波}
\end{chineseabstract}

\begin{englishabstract}
This course project centres on the theme of "Image Enhancement Methods Based on Weber Contrast", aiming to comprehensively apply the image processing theories and programming skills acquired in the Digital Image Processing course to complete a full-scale application case. During the design process, research and exploration were first conducted on the calculation and visualisation methods for Weber contrast. This included employing various filters to isolate backgrounds and presenting the computed Weber contrast through multiple visualisation techniques such as histograms and heatmaps. Subsequently, based on the computed Weber contrast, attempts were made to enhance the image. The results achieved aligned with the expected objectives, validating the effectiveness of the designed algorithm. Finally, attempts were made to extend the algorithm to Michelson contrast and mean square root contrast. However, compared to Weber contrast, the image enhancement effects were inferior. Through this course design, a deeper understanding of image processing theory was gained, and comprehensive abilities in programming implementation and system debugging were enhanced, laying a foundation for more complex practical applications and research in the future.

\englishkeyword{Weber Contrast, Image Enhancement, Michelson Contrast, Root Mean Square Contrast (RMS Contrast), Image Filtering}
\end{englishabstract}

\thesistableofcontents

\chapter{绪\hspace{6pt}论}

\section{课题背景与意义}
随着信息化社会的快速发展,图像已经成为信息获取与传递的重要载体。然而,在图像采集、传输及存储过程中,常常受到环境因素与设备性能的限制,导致结果图像质量不佳、信息模糊。例如,低光照条件下的拍摄环境会导致图像亮度不足、细节缺失;雾霾、雨雪等天气会引入模糊、导致对比度下降;传感器器件的噪声与图像压缩过程中产生的失真则进一步降低了图像质量。这些退化现象不仅影响人类的视觉感知,对计算机视觉系统的后续处理与分析也产生了严重干扰。

在此背景下,图像增强技术应运而生,旨在通过设计算法改善图像的视觉效果,使其符合人眼的感知特性,并增强图像细节,为计算机视觉系统提供更优越的输入数据。

截至目前为止,图像增强技术已在多个领域得到了应用,推动了技术发展。在医学影像中,图像增强技术被用于发现与诊断病灶;在自动驾驶中,增强道路与交通标志的可见性,提高系统的安全性与鲁棒性;在工业质检中,改善缺陷检测的准确性与效率;在深度学习领域,为模型学习提供优质训练数据等等。

综上所述,图像增强技术在理论研究与实际应用中均具有重要意义,不仅改善了图像的视觉表现与信息承载能力,还为医学诊断、智能交通、工业检测等多个关键领域提供了坚实的技术支撑。因此,深入研究图像增强方法,对于推动计算机视觉的发展有着重要价值。\citing{sun2025jiyu}

\section{研究内容与目标}
本课程设计的主要研究内容为“基于 Weber 对比度的图像增强算法”。从最基本的对比度入手,通过研究分析,探索实现基于对比度的图像增强算法。

\subsection{研究目标}
具体的研究目标分解如下:
\begin{enumerate}
    \item 设计算法,计算给定图像的 Weber 对比度,并设计可视化方法,显示计算得到的 Weber 对比度。
    \item 探究不同滤波器下的 Weber 对比度差异,并设计方法融合不同的滤波结果。
    \item 设计算法,基于 Weber 对比度实现图像增强。
    \item 基于先前的研究,尝试扩展到 Michelson 对比度与均值平方根对比度(Root Mean Square Contrast, RMS Contrast)。
\end{enumerate}

\subsection{研究内容}
为实现上述目标,本课程设计将围绕以下核心内容展开:
\begin{enumerate}
    \item Weber 对比度的计算与可视化:
        根据对比度公式进行编程实现,仅针对亮度通道进行计算对比度,随后对结果进行重新着色,并利用直方图等辅助方式对对比度分布进行可视化分析。
        
    \item 滤波器参数影响的对比研究:
        分别构造不同窗宽的高斯滤波器与均值滤波器,以计算对应的 Weber 对比度。直观展示并深入分析不同滤波器尺寸下的结果图像差异,并设计一种图像融合方法将三个不同尺度的结果合并展示。

    \item 基于 Weber 对比度的图像增强算法:
        从 Weber 对比度的计算公式出发进行算法推导,通过设计一个关于原始像素值$I$和局部均值$\mu$的变换函数$T\left(I,\mu\right)$来构造新的$I_s$,以达到增强对比、凸显细节的目的。

    \item 其他对比度的拓展与比较
        尝试将算法拓展到 Michelson 对比度和 RMS 对比度。对比分析并显示三者的结果图像,进行讨论。
\end{enumerate}

\chapter{理论基础与原理分析}

\section{Weber 对比度}
Weber 对比度的定义如下\citing{peli1990contrast}:
\begin{equation}
    \label{weber}
    C_W\left(x,y\right)=\frac{I\left(x,y\right)-I_s\left(x,y\right)}{I_s\left(x,y\right)}
\end{equation}

其中,$\left(x,y\right)$代表图像坐标,$C_W$为该点的 Weber 对比度,$I$为像素灰度,$I_s$为以$\left(x,y\right)$为中心的一个领域的灰度平均值,$I_s\left(x,y\right) = I\left(x,y\right) \circledast w\left(x,y\right)$,$\circledast$表示卷积运算,$w$是低通滤波器。

$C_W\left(x,y\right)$的定义式\ref{weber}中,分子部分的$I\left(x,y\right)-I_s\left(x,y\right)$代表目标像素相对于其局部背景的绝对亮度差异,分母部分为局部背景强度,主要作用是将绝对差异进行归一化。由定义式可知,当$C_W > 0$时,表明该点亮度高于背景,为正对比度,而当$C_W < 0$时,该点亮度低于背景,为负对比度。

图像增强的本质是调整图像的对比度分布,使其更加符合人眼的视觉感知特性。显然, Weber 对比度可以为图像增强提供指导作用,因为它直接反映了各个点的对比度情况,说明基于 Weber 对比度进行图像增强是可行且有效的。

\section{滤波器原理}

\subsection{均值滤波器}
均值滤波器是一种简单的低通滤波器,它在计算局部背景 $I_s$ 时,采用目标像素周围一个 $M \times N$ 窗口内所有像素的算术平均值。对于 $M \times N$ 大小的均值滤波器 $w_{\text{mean}}$,其在离散域的定义为:
$$w_{\text{mean}}(i,j) = \frac{1}{MN}$$
均值滤波器的优点是计算简单、速度快,但在平滑图像时,它对邻域内的所有像素赋予相同的权重,容易导致图像边缘信息模糊。

\subsection{高斯滤波器}
高斯滤波器是另一种常见的低通滤波器,它使用二维高斯函数作为滤波核 $w_{\text{Gaussian}}$。与均值滤波器不同,高斯核的值随着与中心距离的增大而衰减,形成一个高斯曲面 。这使得高斯滤波器在计算局部背景 $I_s$ 时,给予中心像素更高的权重,从而实现了更自然的平滑效果,并能更好地保留图像边缘信息,这与人眼视觉系统的权重分布特性更加接近。

二维高斯滤波器的核 $w_{\text{Gaussian}}$ 在连续域的定义如下:
$$G(x,y;\sigma) = \frac{1}{2\pi\sigma^2} e^{-\frac{x^2+y^2}{2\sigma^2}}$$
其中,标准差 $\sigma$ (Sigma) 是高斯滤波器的关键参数,它决定了滤波器的平滑程度和局部背景 $I_s$ 所代表的空间尺度。$\sigma$ 越大,滤波核越宽,平滑效果越强,即局部背景的范围越大。

\section{其他对比度定义}
\begin{description}
    \item [Michelson 对比度] 
        \begin{equation}
            C_\textit{Michelson} = \frac{I_\text{max}-I_\text{min}}{I_\text{max}+I_\text{min}}
            \label{eq:m_def}
        \end{equation}
        式中,$I_{\text{max}}$ 和 $I_{\text{min}}$ 分别是 $\left(x,y\right)$ 为中心的像素的某个邻域中的像素最大值和最小值。
        
    \item [RMS 对比度]
        \begin{equation}
            C_\textit{RMS} = \sqrt{\frac{1}{M_\Omega}\sum_\Omega \left(\frac{I-\bar{I}}{\bar{I}}\right)^2}
            \label{eq:r_def}
        \end{equation}
        式中,$C_{\text{RMS}}$ 为均方根对比度,$I$ 是像素灰度,$\bar{I}$ 是 $\left(x,y\right)$ 的邻域内像素灰度的平均值,$\Omega$ 是以 $\left(x,y\right)$ 为中心的邻域,$M_{\Omega}$ 是邻域内像素总数。
\end{description}

\chapter{Weber 对比度提取与可视化实现}
\label{cha:3}

实验中使用的图片如图\ref{fig:test}所示,使用窗宽为 71x71 的高斯滤波器。

\begin{figure}[h]
    \centering
    \includegraphics[width=0.8\textwidth]{pic/test_img.pdf}
    \caption{测试图像}
    \label{fig:test}
\end{figure}

\section{提取算法设计}
首先,对原图像进行滤波得到背景亮度 $I_s$。在获得背景亮度后,Weber 对比度的实际计算公式如下:
\begin{equation}
    \label{actual_weber}
    C_W\left(x,y\right) = \frac{I\left(x,y\right) - I_s\left(x,y\right)}{I_s\left(x,y\right) + \epsilon}
\end{equation}

与式\ref{weber}不同,此处在分母上添加了极小值 $\epsilon$,用于防止分母为零时导致计算错误。该公式反映了当前像素相对于局部背景的相对变化率,能够有效抑制光照带来的低频干扰,突出图像的高频细节。

计算得到 Weber 对比度后,利用 \texttt{summary} 函数提取描述性统计量(最小值、中位数、最大值等),为后续增强算法的参数优化提供定量依据。从表\ref{tab:summary_stats}中 Weber 对比度的统计数据可以看出,其取值范围分布较广,最小值达到 $-1.0000$,最大值则高达 $13.8949$,这表明图像中既存在比背景暗得多的区域,也存在显著亮于背景的细节特征。中位数($-0.1083$)和平均值($-0.0599$)均略小于零,反映出图像中大部分区域的亮度略低于局部背景均值。此外,较大的标准差($0.5831$)体现了对比度分布的离散性,为后续通过非线性变换增强图像细节提供了充足的调整空间。
\begin{table}[htbp]
    \centering
    \begin{tabular}{l c c c}
        \toprule
        \textbf{统计特征} & \textbf{Weber} & \textbf{Michelson} & \textbf{RMS} \\
        \midrule
        最小值 (Min) & $-1.0000$ & $0.0000$ & $0.0000$ \\
        中位数 (Median) & $-0.1083$ & $0.6471$ & $0.2681$ \\
        最大值 (Max) & $13.8949$ & $1.0000$ & $1.9457$ \\
        平均值 (Mean) & $-0.0599$ & $0.6184$ & $0.2980$ \\
        标准差 (Std Dev) & $0.5831$ & $0.2982$ & $0.1989$ \\
        \bottomrule
    \end{tabular}
    \caption{各个对比度的统计特征}
    \label{tab:summary_stats}
\end{table}

\section{对比度可视化方法}

\subsection{直方图}
\label{ssec:histo}
直方图作为一维概率密度函数的离散近似表示,是图像统计特征分析的重要工具\citing{gonzalez2002digital}。对于 Weber 对比度 $C_W(x,y)$,其直方图定义为:
\begin{equation}
    h(r_k) = n_k
\end{equation}
其中,$r_k$ 表示第 $k$ 个离散对比度级别,$n_k$ 为图像中对比度值等于 $r_k$ 的像素总数。归一化直方图可表示为:
\begin{equation}
    p(r_k) = \frac{n_k}{MN}
\end{equation}
式中,$M \times N$ 为图像尺寸,$p(r_k)$ 近似表示对比度值为 $r_k$ 的概率。

通过直方图分析可获取以下统计特征:
\begin{description}
    \item[动态范围] 对比度值域 $[C_{\min}, C_{\max}]$ 反映图像局部亮度变化的幅度,动态范围越大表明图像包含更丰富的细节层次;
    \item[对称性分析] 正对比度($C_W > 0$)占比 $P_+ = \sum_{r_k>0} p(r_k)$ 与负对比度($C_W < 0$)占比 $P_- = \sum_{r_k<0} p(r_k)$ 的相对关系,反映图像整体的亮度分布特征;
\end{description}

本课程设计采用 MATLAB 的 \texttt{histogram} 函数构建直方图,结果如图\ref{fig:histo_w}所示。从直方图中可以观察到,Weber 对比度的数值主要分布在 $-0.5$ 到 $+2$ 的区间内,并且在零点附近出现了峰值,这与表\ref{tab:summary_stats}中的统计特征相符。结合表\ref{tab:summary_stats}中指示的 Weber 对比度最大值,可以预见,如果直接将 Weber 对比度作为图像显示,由于动态范围过大,最终的图像可能会出现过暗的情况。我们在图\ref{fig:w_direct}中可以看到这种效果。

\begin{figure}[hbtp]
    \centering
    \includegraphics[width=0.8\textwidth,trim=3cm 10cm 3cm 10cm,clip]{pic/weber_histo.pdf}
    \caption{Weber 对比度直方图}
    \label{fig:histo_w}
\end{figure}

\subsection{热力图}
热力图(Heatmap)作为二维标量场可视化的有效手段,通过伪彩色映射技术将数值矩阵转换为符合人眼感知特性的色彩分布图\citing{ware2012information}。对于 Weber 对比度场 $C_W: \Omega \rightarrow \mathbb{R}$(其中 $\Omega \subset \mathbb{R}^2$ 为图像空间域),伪彩色映射可表示为:
\begin{equation}
    \mathbf{C}(x,y) = \mathcal{M}\left(\frac{C_W(x,y) - C_{\min}}{C_{\max} - C_{\min}}\right)
\end{equation}
其中,$\mathcal{M}: [0,1] \rightarrow \mathbb{R}^3$ 为色彩映射函数(Colormap),将归一化的对比度值映射到 RGB 三维色彩空间。

热力图可视化的实现流程与代码实现保持一致,具体为:
\begin{enumerate}
    \item 直接以 \texttt{imagesc} 对 $C_W \in \mathbb{R}^{M \times N}$ 做自适应尺度映射,函数内部使用当前数据的最小、最大值完成线性拉伸并进行插值着色;
    \item 选用感知均匀的 \texttt{pink} 色图以维持色彩单调性与可辨性;
    \item 添加 \texttt{colorbar} 建立数值—色彩对应关系,便于定量读取。
\end{enumerate}

热力图可视化保留了 Weber 对比度的空间拓扑结构,能够直观揭示图像中边缘、纹理等高频成分的空间分布特征。高对比度区域($|C_W| \gg 0$)对应图像的显著性特征,而低对比度区域($|C_W| \approx 0$)则表征平坦的背景区域,结果如图\ref{fig:w_heatmap}。

\begin{figure}
    \includegraphics[width=0.8\textwidth, trim=3cm 10cm 3cm 10cm, clip]{./pic/weber_heatmap.pdf}
    \caption{Weber 对比度热力图}
    \label{fig:w_heatmap}
\end{figure}

\subsection{可视化 Weber 对比度}
Weber 对比度 $C_W(x,y)$ 作为无量纲相对量,且分布非均匀。为直观展示其空间分布,需构建从对比度空间到图像灰度空间的映射函数,再构建从灰度空间到图像彩色空间的映射函数。

\subsubsection{构建从灰度空间到彩色空间的映射函数}
Weber 对比度的计算在灰度空间进行,为了更加直观地观察对比度分布及增强效果,需要将可视化用的灰度结果映射回彩色空间。本课程设计采用的映射策略为:将原始彩色图像转换至 HSV 色彩空间,随后用可视化得到的灰度图像替换 V(明度)通道,再逆变换回 RGB 色彩空间。

\subsubsection{线性缩放得到可视化图像}
首先对 Weber 对比度采用 \texttt{mat2gray} 函数进行线性缩放,将其映射至 $[0, 1]$ 区间,得到初步的可视化图像,如图\ref{fig:w_direct}所示。与第\ref{ssec:histo}部分的分析一致,可视化结果整体亮度较暗,需要进一步的处理以改善视觉效果。本研究采用了两种不同的增强方法,分别为伽马变换与直方图均衡化,增强结果分别如图\ref{fig:w_gamma}和图\ref{fig:w_histoeq}所示。

\begin{figure}[hbtp]
    \centering
    \includegraphics[width=0.8\textwidth, trim=0cm 7cm 0cm 7cm, clip]{./pic/weber_direct.pdf}
    \caption{Weber 对比度直接可视化结果}
    \label{fig:w_direct}
\end{figure}

\subsubsection{对可视化结果实施伽马变换}
伽马变换(Gamma Correction)源于显示设备的非线性光电响应特性建模,现已广泛应用于图像的感知亮度调整\citing{poynton2012digital}。其数学形式为幂律变换:
\begin{equation}
    \label{eq:gamma}
    I_{\text{out}}(x,y) = I_{\text{in}}(x,y)^{\gamma}, \quad I_{\text{in}}, I_{\text{out}} \in [0,1]
\end{equation}
其中,$\gamma$ 为伽马指数。根据韦伯-费希纳定律(Weber-Fechner Law)\citing{fechner1860elemente},人眼对亮度的感知遵循对数关系,即感知亮度 $L_p$ 与物理亮度 $L$ 满足:
\begin{equation}
    L_p = k \log L + C
\end{equation}
伽马变换通过幂函数近似该对数响应,当 $\gamma < 1$ 时,变换曲线上凸,扩展低亮度区间的动态范围;当 $\gamma > 1$ 时,曲线下凹,压缩高亮度区间。

对于归一化 Weber 对比度 $\tilde{C}_W \in [0,1]$,应用伽马校正:
\begin{equation}
    \hat{C}_W(x,y) = \left[\tilde{C}_W(x,y)\right]^{\gamma}
\end{equation}

本研究设定 $\gamma = 0.6$,该参数选择基于以下考量:
\begin{itemize}
    \item 根据对比度直方图分布特征,低对比度像素占比较高,需通过 $\gamma < 1$ 扩展其表示空间;
    \item 经验性地平衡增强效果与噪声放大风险,避免过度增强导致的伪影。
\end{itemize}

变换后的对比度 $\hat{C}_W$ 替换 HSV 空间的明度分量,经逆变换得到感知优化的可视化图像 $\mathbf{I}_{wg}$(图\ref{fig:w_gamma})。伽马变换的优势在于计算复杂度低(单点操作,复杂度 $O(MN)$),且能够依据人类视觉系统的非线性响应特性实现亮度的感知均衡化。

由于 Weber 对比度分布的非均匀性较为显著,经伽马变换后的结果(如图\ref{fig:w_gamma}所示)仍未达到理想效果。虽然亮度有所改善,但整体图像依然偏暗,难以充分展现对比度信息的细节特征。

\subsubsection{对可视化结果实施直方图均衡化}
直方图均衡化(Histogram Equalization, HE)是基于概率论的全局对比度增强技术,其理论基础是通过累积分布函数(Cumulative Distribution Function, CDF)变换实现概率密度的均匀化\citing{pizer1987adaptive}。

设归一化 Weber 对比度 $\tilde{C}_W$ 的概率密度函数为 $p_C(r)$,其累积分布函数定义为:
\begin{equation}
    F_C(r) = \int_0^r p_C(\xi) \mathrm{d}\xi = P(\tilde{C}_W \leq r)
\end{equation}

直方图均衡化通过变换函数 $s = T(r) = F_C(r)$ 将原始分布映射至均匀分布。根据概率论中的随机变量变换定理,若 $s = T(r)$ 单调递增且可微,则变换后的概率密度函数为:
\begin{equation}
    p_S(s) = p_C(r) \left| \frac{\mathrm{d}r}{\mathrm{d}s} \right| = p_C(r) \cdot \frac{1}{p_C(r)} = 1, \quad s \in [0,1]
\end{equation}
即输出 $s$ 服从均匀分布 $\mathcal{U}(0,1)$,从而最大化图像熵:
\begin{equation}
    H(S) = -\int_0^1 p_S(s) \log p_S(s) \mathrm{d}s = -\int_0^1 1 \cdot \log 1 \, \mathrm{d}s = 0 \rightarrow \text{最大不确定性}
\end{equation}

在离散实现中,对于 $L$ 级灰度图像,均衡化变换采用离散累积直方图:
\begin{equation}
    s_k = T(r_k) = \sum_{j=0}^{k} p_C(r_j) = \sum_{j=0}^{k} \frac{n_j}{MN}, \quad k=0,1,\ldots,L-1
\end{equation}

应用于 Weber 对比度可视化时,算法流程为:
\begin{enumerate}
    \item 对 $\tilde{C}_W$ 构建归一化直方图 $\{p_C(r_k)\}_{k=0}^{L-1}$;
    \item 计算累积分布函数 $\{F_C(r_k)\}$ 并作为映射表;
    \item 通过查找表(LUT)变换得到均衡化对比度 $\breve{C}_W(x,y) = F_C[\tilde{C}_W(x,y)]$;
    \item 以 $\breve{C}_W$ 替换 HSV 明度分量,逆变换得 $\mathbf{I}_{wh}$。
\end{enumerate}

直方图均衡化的优势在于无需手动参数调节,具有自适应性,其目标函数隐式地最大化对比度的变异程度:
\begin{equation}
    \max_{T} \, \mathbb{E}[|T(C_W(x,y)) - T(C_W(x',y'))|]
\end{equation}

结果如图\ref{fig:w_histoeq}所示。与伽马变换相比,直方图均衡化的增强效果更为显著。图像整体偏暗的问题得到了有效改善,对比度分布更加均衡,视觉效果得到优化。然而,图像四周仍存在轻微的失真现象,这主要源于两个因素:其一是原图中暗区像素,其二是对比度动态范围过大引起在缩放过程中的信息损失。虽然直方图均衡化已初步实现了图像增强的效果,但仍存在进一步改进的空间。我们将在\ref{cha:4}展开讨论。

\begin{figure}[hbtp]
    \centering
    \subfloat[伽马变换结果]{
        \includegraphics[width=0.45\textwidth, trim=0cm 7cm 0cm 5cm, clip]{./pic/weber_gamma.pdf}
        \label{fig:w_gamma}
    }
    \hfill
    \subfloat[直方图均衡化结果]{
        \includegraphics[width=0.45\textwidth, trim=0cm 7cm 0cm 5cm, clip]{./pic/weber_histoeq.pdf}
        \label{fig:w_histoeq}
    }
    \caption{Weber 对比度可视化的增强方法对比}
\end{figure}


\section{不同滤波器下的 Weber 对比度可视化结果}

\subsection{滤波器对比}
Weber 对比度的核心在于对局部背景 $I_s$ 的估计。背景的计算本质上是一种低通滤波,滤波器类型与参数直接决定了 $I_s$ 的空间尺度与平滑特性,从而影响 $C_W$ 的数值分布与可视化结果的观感。在本设计中,分别采用了均值滤波器与高斯滤波器来构造背景:
\begin{equation}
    I_s^{\mathrm{mean}}(x,y) = I(x,y) \circledast w_{\mathrm{mean}}, \quad w_{\mathrm{mean}}(i,j)=\frac{1}{MN}
\end{equation}
\begin{equation}
    I_s^{\mathrm{Gauss}}(x,y) = I(x,y) \circledast G(\sigma), \quad G(x,y;\sigma)=\frac{1}{2\pi\sigma^2}\exp\!\left(-\frac{x^2+y^2}{2\sigma^2}\right)
\end{equation}
两者的关键差异在于权重分布:均值滤波对邻域内像素赋予相同权重,边缘处易出现显著的模糊与“阶跃平滑”现象;高斯滤波则随与中心距离增大而权重递减,更接近人眼视觉系统的空间权重特性,因而在保留边缘与纹理方面更占优势。依据式\ref{actual_weber},在采用相同窗宽时,$I_s^{\mathrm{Gauss}}$ 通常能在不显著引入伪影的前提下抑制噪声并保持细节,可视化结果在整体亮度与细节显著性之间呈现更合理的平衡;而 $I_s^{\mathrm{mean}}$ 的可视化结果更倾向于全局平滑,细微结构的对比度被抑制,边缘过渡变钝。综合考虑鲁棒性与感知质量,高斯滤波更适合作为 Weber 对比度的背景估计器。

\subsection{窗宽对比}
窗宽(或 $\sigma$)决定了背景估计的空间尺度。本设计选取了三种代表性的尺度 $w\in\{5,\,71,\,351\}$,并令 $\sigma= \lfloor (w-1)/6 \rfloor$ 保持高斯核覆盖范围与窗宽的对应关系。不同尺度下的典型表现可概括如下:
\begin{description}
    \item[小尺度($w=5$)] 背景高度局部化,平坦区域的 $C_W$ 接近 0,边缘与细纹理处呈现窄而尖的响应峰;可视化结果在细节处对比强,但对噪声也更敏感,暗区可能出现颗粒感;适用于强化细节的场景,但需配合去噪策略。
    \item[中尺度($w=71$)] 在噪声抑制与边缘保留之间取得较好平衡,是默认参数选择;$C_W$ 的动态范围适中,可视化结果整体亮度更稳定,细节显著性与视觉舒适度兼顾,适合一般场景。
    \item[大尺度($w=351$)] 背景近似全局照度场,$C_W$ 在大结构与缓慢渐变区域的相对变化更显著;但在强边缘附近可能出现环状伪影与过增强区域,可视化结果容易产生主观不适,应谨慎使用或结合后处理(如边缘抑制、引导滤波)。
\end{description}

\begin{figure}[hbtp]
    \centering
    \subfloat[$5x5$]{
        \includegraphics[width=0.3\textwidth, trim=0cm 7cm 0cm 5cm, clip]{./pic/w_m_5.pdf}
        \label{fig:w_m5}
    }
    \subfloat[$71x71$]{
        \includegraphics[width=0.3\textwidth, trim=0cm 7cm 0cm 5cm, clip]{./pic/w_m_71.pdf}
        \label{fig:w_m71}
    }
    \subfloat[$351x351$]{
        \includegraphics[width=0.3\textwidth, trim=0cm 7cm 0cm 5cm, clip]{./pic/w_m_351.pdf}
        \label{fig:w_m351}
    }
    \caption{根据不同窗宽的均值滤波器得到的 Weber 对比度可视化图像}
    \label{fig:w_m}
\end{figure}

\begin{figure}[hbtp]
    \centering
    \subfloat[$5x5$]{
        \includegraphics[width=0.3\textwidth, trim=0cm 7cm 0cm 5cm, clip]{./pic/w_g_5.pdf}
        \label{fig:w_g5}
    }
    \subfloat[$71x71$]{
        \includegraphics[width=0.3\textwidth, trim=0cm 7cm 0cm 5cm, clip]{./pic/w_g_71.pdf}
        \label{fig:w_g71}
    }
    \subfloat[$351x351$]{
        \includegraphics[width=0.3\textwidth, trim=0cm 7cm 0cm 5cm, clip]{./pic/w_g_351.pdf}
        \label{fig:w_g351}
    }
    \caption{根据不同窗宽的高斯滤波器得到的 Weber 对比度可视化图像}
    \label{fig:w_g}
\end{figure}

\subsection{不同窗宽下的滤波结果的融合显示}
为直观呈现不同尺度下的背景估计差异,采用“尺度融合”的伪彩显示方法:将同一原图在三种窗宽下的滤波结果分别映射到 RGB 三通道,实现多尺度信息的联合展示。设滤波器族 $\mathcal{W}=\{w_1, w_2, w_3\}$,对应的背景估计为 $\{I_s^{(w_1)}, I_s^{(w_2)}, I_s^{(w_3)}\}$,则融合图像可表示为
\begin{equation}
    \mathbf{F}_{\mathrm{mean}}(x,y) = \big[I\circledast w_{\mathrm{mean}}(w_1),\ I\circledast w_{\mathrm{mean}}(w_2),\ I\circledast w_{\mathrm{mean}}(w_3)\big]
\end{equation}
\begin{equation}
    \mathbf{F}_{\mathrm{Gauss}}(x,y) = \big[I\circledast G(\sigma_1),\ I\circledast G(\sigma_2),\ I\circledast G(\sigma_3)\big],\ \ \sigma_i=\left\lfloor \frac{w_i-1}{6} \right\rfloor
\end{equation}
其中,约定 $R$、$G$、$B$ 通道分别对应小、中、大尺度。该显示在感知上的解读为:
\begin{itemize}
    \item 红色成分突出边缘与细纹理,反映小尺度结构的背景变化;
    \item 绿色成分描述中尺度的纹理与形状特征,兼顾细节与均衡;
    \item 蓝色成分体现大尺度的照度与缓慢渐变背景,是全局光照的指示器。
\end{itemize}
通过图\ref{fig:w_fm}和图\ref{fig:w_fg}对比 $\mathbf{F}_{\mathrm{mean}}$ 与 $\mathbf{F}_{\mathrm{Gauss}}$,可观察到前者的边缘颜色过渡更突兀、细节显得“块状”,而后者在同尺度下更为平滑且层次自然。这一融合视图有助于参数选取与直觉验证:当红通道过强且伴随噪声纹理,应适当增大窗宽或改用高斯核;当蓝通道主导且细节被压制,则应缩小窗宽或引入非线性增强(如伽马校正、对比度受限的自适应直方图均衡化)。综合实验观察,采用高斯滤波、窗宽 $w=71$的融合结果在边缘保留与视觉舒适度方面更具优势,也与基于 Weber 对比度的可视化图达到较好的主观一致性。

\begin{figure}[hbtp]
    \includegraphics[width=0.8\textwidth, trim=0cm 7cm 0cm 5cm, clip]{./pic/weber_fuse_m.pdf}
    \caption{均值滤波器融合结果}
    \label{fig:w_fm}
\end{figure}

\begin{figure}[hbtp]
    \includegraphics[width=0.8\textwidth, trim=0cm 7cm 0cm 7cm, clip]{./pic/weber_fuse_g.pdf}
    \caption{高斯滤波器融合结果}
    \label{fig:w_fg}
\end{figure}

\chapter{基于 Weber 对比度的图像增强算法}
\label{cha:4}
\section{增强原理设计}

基于式\ref{actual_weber}得到的 Weber 对比度 $C_W(x,y)$,我们设计一个同时考虑局部对比度的非线性增强项与背景照度的感知校正项的组合变换,以兼顾细节强化与视觉舒适度:
\begin{equation}
    \label{eq:enh_transform}
    I_{\text{out}}(x,y) = I_s(x,y)^{\gamma}\,\Big(1 + \beta\cdot\operatorname{sgn}(C_W)\cdot |C_W(x,y)|^{\alpha}\Big), \quad \gamma>0,\ \alpha>0,\ \beta\in(0,1]
\end{equation}
其中 $I_s$ 为局部背景(低通分量),$\gamma$ 为感知亮度校正指数,$\alpha$ 为对比度的非线性强化指数,$\beta$ 为强度缩放因子。该设计遵循以下原则:
\begin{itemize}
    \item \textbf{感知一致性($\gamma$)}:$I_s^{\gamma}$ 模拟显示体系与人眼的非线性响应(参见式\ref{eq:gamma}与文献\citing{poynton2012digital}),$\gamma<1$ 时扩展暗区动态范围、抑制过曝;$\gamma>1$ 时压缩高亮区域。
    \item \textbf{边缘/纹理强化($\alpha$)}:$|C_W|^{\alpha}$ 为幂律抑制小对比度、相对提升中高对比度的非线性映射,$\alpha\in(0,1]$ 有助于降低噪声放大风险。
    \item \textbf{强度可控($\beta$)}:缩放系数 $\beta$ 保证增强幅度受控,便于在不同场景下平衡细节显著性与伪影风险。
\end{itemize}
为保证输出的物理可行性与显示稳定性,对 $I_{\text{out}}$ 施加区间裁剪:
\begin{equation}
    \label{eq:clip}
    I_{\text{out}} \leftarrow \min\big\{1,\ \max\{0,\ I_{\text{out}}\}\big\},\quad I\in[0,1]\ \Rightarrow\ I_{\text{out}}\in[0,1]
\end{equation}
实际显示采用 HSV 映射:将原始彩色图的 HSV 表示中 V 通道替换为 $I_{\text{out}}$,再逆变换到 RGB 空间。该策略保留原色相与饱和度,仅调整明度,从而避免色彩失真。

\section{算法实现步骤}

结合\ref{cha:3}章的处理流程,完整的增强算法如下:
\begin{enumerate}
    \item \textbf{读取与预处理}:读取彩色图像并获得灰度 $I\in[0,1]$,保留原彩色图用于后续 HSV 映射。
    \item \textbf{背景估计}:选取滤波器与参数(推荐高斯核,窗宽 $w=71$,$\sigma=\lfloor(w-1)/6\rfloor$),计算 $I_s=I\circledast G(\sigma)$。
    \item \textbf{对比度计算}:按式\ref{actual_weber}计算 $C_W=(I-I_s)/(I_s+\epsilon)$,并提取统计量用于参数参考(参见表\ref{tab:summary_stats})。
    \item \textbf{非线性对比度映射}:$C_W^{\text{new}}=\beta\cdot\operatorname{sgn}(C_W)\cdot |C_W|^{\alpha}$,其中 $\alpha=0.8$、$\beta=0.85$ 为经验值。
    \item \textbf{照度感知校正}:$I_s^{\text{new}}=I_s^{\gamma}$,推荐 $\gamma=0.4$ 以扩展暗区动态范围。
    \item \textbf{合成与裁剪}:按式\ref{eq:enh_transform}得到 $I_{\text{out}}=I_s^{\text{new}}\cdot(1+C_W^{\text{new}})$,再依式\ref{eq:clip}裁剪至 $[0,1]$。
    \item \textbf{颜色恢复与显示}:将原图转换到 HSV,将 V 通道替换为 $I_{\text{out}}$,逆变换得到增强后的彩色图像。
\end{enumerate}

复杂度方面,核心为卷积与逐点幂律/乘加运算,总体为 $O(MN)$(固定核大小)。参数选择可参照 $C_W$ 的直方图与统计量:当暗区占比高且噪声偏大,取较小 $\beta$ 与较大 $\gamma$;当边缘不够显著,适当提高 $\beta$ 或减小 $\alpha$ 的抑制力度。

\section{增强结果与对比分析}

在典型自然场景与低照度图像上,采用 $w=71$、$\sigma=\lfloor(w-1)/6\rfloor$、$\alpha=0.8$、$\beta=0.85$、$\gamma=0.4$ 的配置具有较好的通用性。结果如图\ref{fig:w_res},综合表现如下:
\begin{itemize}
    \item \textbf{暗区改善}:$\gamma<1$ 扩展了 V 通道的暗部动态范围,细节显著提升,整体亮度更均衡(与第\ref{ssec:histo}节直方图分析一致)。
    \item \textbf{边缘与纹理}:$|C_W|^{\alpha}$ 强化了中高对比度结构,边缘更清晰、纹理更具层次;相较直接线性缩放,噪声放大显著降低。
    \item \textbf{色彩稳定}:HSV 的 V 替换策略避免了色相偏移,主观色彩更稳定。
\end{itemize}

与\ref{cha:3}章的两种可视化增强(伽马变换与全局直方图均衡化)相比,式\ref{eq:enh_transform} 的组合模型在保留细节、控制噪声与维持整体亮度方面更为平衡,同时,在图\ref{fig:w_histoeq}中出现的块状失真也得到了显著改善。本方法以局部对比度为导向、以背景照度为锚点进行调节,能够更贴合人眼对亮度与对比度的联合感知特性。此外,该模型参数具有明确的物理/感知含义,便于根据直方图与场景内容进行快速调优。

\begin{figure}[hbt]
    \includegraphics[width=\textwidth, trim=0cm 7cm 0cm 7cm, clip]{./pic/weber_res.pdf}
    \caption{使用 Weber 对比度进行图像增强的结果}
    \label{fig:w_res}
\end{figure}

\chapter{扩展性讨论}

\section{扩展到 Michelson 对比度的可行性}

Michelson 对比度强调局部极值比值,适合周期纹理或亮暗交替场景;但对自然图像的鲁棒性较弱,特别是在光照不均与噪声存在时。基于\ref{eq:m_def},我们进行了计算与增强尝试,增强效果整体不及 Weber 对比度,且易出现块状伪影。

\subsection{计算与可视化}
采用窗宽 $w\!\times\!w$(实验取 $7\!\times\!7$)的滑动窗口,通过 \texttt{ordfilt2} 计算局部 $I_{\max}$ 与 $I_{\min}$,并以
\begin{equation}
    C_{\mathrm{Michelson}} = \frac{I_{\max}-I_{\min}}{I_{\max}+I_{\min}+\epsilon}
\end{equation}
得到对比度图。直方图与热力图分别如图\ref{fig:m_histo}和图\ref{fig:m_heat}所示,$C_{\mathrm{Michelson}}$ 分布在 $[0,1]$,在均匀区域趋近 0,在强边缘附近接近 1,但对噪声与孤立亮点较敏感。

\begin{figure}[hbtp]
    \centering
    \includegraphics[width=0.8\textwidth, trim=0cm 7cm 0cm 7cm, clip]{./pic/m_histo.pdf}
    \caption{Michelson 对比度直方图}
    \label{fig:m_histo}
\end{figure}

\begin{figure}[hbtp]
    \centering
    \includegraphics[width=0.8\textwidth, trim=0cm 7cm 0cm 7cm, clip]{./pic/m_heatmap.pdf}
    \caption{Michelson 对比度热力图}
    \label{fig:m_heat}
\end{figure}

\subsection{增强思路与实现}
与 Weber 框架不同,此处直接在局部块内调节亮暗动态范围:设目标对比度 $C_{\mathrm{target}}=0.8$,由 $C=(k-1)/(k+1)$ 得目标极值比 $k\approx 9$。代码以滑动窗口构造列块矩阵,选取 $C_m<0.6$ 且 $I_{\max}\neq I_{\min}$ 的块进行线性拉伸,使 $I_{\max}/I_{\min}$ 接近目标比值,再整体施加 $\gamma=0.6$ 的幂律校正并映射回彩色图(HSV 仅替换 V)。

需要指出,代码中对重叠块未做加权平均,是因为经验证该平均策略在本场景下效果不佳,故保留单块写回并通过窗口较小化减轻重叠现象。即便如此,相比 Weber 对比度,Michelson 对比度的“极值驱动”在复杂纹理处容易放大噪声。

\subsection{结果与讨论}
增强结果如图\ref{fig:m_res}。相较于 Weber(图\ref{fig:w_res}),Michelson 模型在:
\begin{itemize}
    \item \textbf{细节与亮暗平衡}:局部极值拉伸带来一定细节提升,但暗区与高光区域易出现不均匀增益,整体亮度平衡欠佳,仍需伽马变换调整对比度;
    \item \textbf{伪影与噪声}:块状拼接痕迹与局部过冲更明显,尤其是在图像的眼睛部分,可以看到明显的重叠现象;噪声点被放大,需额外的边缘/噪声抑制;
    \item \textbf{主观舒适度}:相比 Weber 的平滑且可控增强,Michelson 的效果在自然场景下缺乏稳定性。
\end{itemize}

综上,Michelson 对比度在本实验的自然图像场景中不如 Weber 稳定,适合作为补充特征或用于结构规整、周期性纹理的特殊场景;若需进一步使用,建议引入重叠块加权融合、引导滤波或双边滤波以缓解伪影,并降低目标对比度阈值以控制过冲。

\begin{figure}[hbt]
    \centering
    \subfloat[直接可视化]{\includegraphics[width=0.45\textwidth, trim=0cm 7cm 0cm 7cm, clip]{./pic/m_direct.pdf}}
    \hfill
    \subfloat[增强结果]{\includegraphics[width=0.45\textwidth, trim=0cm 7cm 0cm 7cm, clip]{./pic/m_res.pdf}}
    \caption{使用 Michelson 对比度进行图像增强的结果}
    \label{fig:m_res}
\end{figure}

\section{扩展到均值平方根对比度的可行性}

RMS 对比度衡量局部强度偏差的均方根,天然对极端值不敏感,适合作为纹理强度的整体度量;但其对边缘的定向性较弱,增强时不如 Weber 针对性强。基于式\ref{eq:r_def} 和 \texttt{rms.m} 的实现,实验发现其增强效果仍不及 Weber,更多地表现为整体对比度提振而非细节强化。

\subsection{计算与可视化}
采用 $7\!\times\!7$ 平均核 $h$ 获取局部均值 $I_{\text{ave}}$,按
\begin{equation}
    C_{\mathrm{RMS}} = \sqrt{\,h \circledast \bigg(\Big(\frac{I-I_{\text{ave}}}{I_{\text{ave}}+\epsilon}\Big)^2\bigg)}
\end{equation}
得到 RMS 对比度图。直方图与热力图见图\ref{fig:r_histo_heat},分布集中在 $[0, 1.2]$,整体较 Michelson 更平滑,对孤立噪声的敏感度较低。

\begin{figure}[hbtp]
    \centering
    \subfloat[RMS 对比度直方图]{\includegraphics[width=0.45\textwidth, trim=0cm 7cm 0cm 7cm, clip]{./pic/rms_histo.pdf}}
    \hfill
    \subfloat[RMS 对比度热力图]{\includegraphics[width=0.45\textwidth, trim=0cm 7cm 0cm 7cm, clip]{./pic/rms_heatmap.pdf}}
    \caption{RMS 对比度的直方图与热力图}
    \label{fig:r_histo_heat}
\end{figure}

\subsection{增强思路与实现}
代码先将 $C_{\mathrm{RMS}}$ 归一化到 $[0,1]$,构造增益图
\begin{equation}
    G = 1 + \alpha\,(1-C_{\mathrm{RMS}})^{\beta}, \quad \alpha=1.2,\ \beta=1.6
\end{equation}
并用高斯核($\sigma=1$)对 $G$ 平滑以减少块状变化。随后以
\begin{equation}
    I_e = I_{\text{ave}} + G\cdot (I - I_{\text{ave}}), \quad I_e\in[0,1]
\end{equation}
重建灰度,最终在 HSV 中替换 V 通道(附加 $\gamma=0.6$ 伽马变换)。

\subsection{结果与讨论}
结果如图\ref{fig:r_res}。相较 Weber 对比度:
\begin{itemize}
    \item \textbf{亮度均衡}:全局对比度有所提升,暗区亮度改善,但细节锐度不足;
    \item \textbf{边缘表现}:因 RMS 对极值不敏感,边缘强化有限,纹理对比不如 Weber 明显;
    \item \textbf{伪影控制}:高斯平滑的增益图抑制了块状伪影,噪声放大相对可控;但在高频区域仍可能出现轻微“雾化”。
\end{itemize}

总体看,RMS 对比度适合作为全局/半局部的对比度指标,用于温和提升整体对比,而非强调精细结构。在自然场景下仍弱于 Weber;若继续使用,可调低 $\alpha$、提高平滑 $\sigma$ 以进一步抑制噪声,或结合边缘引导的增益约束以增强细节。

\begin{figure}[hbt]
    \centering
    \subfloat[直接可视化]{\includegraphics[width=0.45\textwidth, trim=0cm 7cm 0cm 7cm, clip]{./pic/rms_direct.pdf}}
    \hfill
    \subfloat[增强结果]{\includegraphics[width=0.45\textwidth, trim=0cm 7cm 0cm 7cm, clip]{./pic/rms_res.pdf}}
    \caption{使用 RMS 对比度进行图像增强的结果}
    \label{fig:r_res}
\end{figure}

\chapter{结\hspace{6pt}论}

\section{工作总结}

本课程设计按照预定目标,完整地实现了四个主要研究内容:

首先,成功设计了 Weber 对比度的计算与可视化方案。通过直方图、热力图、线性缩放、伽马变换及直方图均衡化等多种方法,对对比度分布进行了详细分析,为后续增强奠定了基础。

其次,系统研究了滤波器类型与窗宽对对比度的影响。对比实验表明,高斯滤波器相比均值滤波在边缘保留与视觉舒适度上更具优势;中等窗宽($w=71$)在噪声控制与细节保留间取得较好平衡,是通用的参数选择。设计的三尺度融合显示方法直观展现了多尺度结构,便于直观理解与参数调优。

再次,设计并实现了基于 Weber 对比度的图像增强算法。通过组合非线性对比度映射与背景照度感知校正,在式\ref{eq:enh_transform}中统一了对比度强化与亮度校正,实验表明该方法在暗区改善、边缘清晰度、色彩稳定性及伪影抑制方面均优于直接线性缩放与全局直方图均衡化。

最后,尝试了 Michelson 和 RMS 对比度的扩展。结果表明,虽然这两种对比度各有特点,但在自然图像场景下的增强效果不如 Weber,更适合作为补充特征或用于特定类型的图像处理任务。

\section{主要创新与贡献}

本设计的主要创新点包括:
\begin{enumerate}
    \item \textbf{组合增强模型}:式\ref{eq:enh_transform}同时考虑局部对比度非线性映射与背景照度感知校正,参数具有明确的物理与感知含义,相比全局线性增强更加精细且稳定。
    \item \textbf{多尺度融合显示}:通过 RGB 三通道映射三种窗宽的滤波结果,提供了直观的参数选择与调试工具,简洁有效。
    \item \textbf{完整对比体系}:在同一框架下对三种对比度进行了系统实验与对比分析,为不同应用场景下对比度指标的选择提供了参考。
\end{enumerate}

从实践角度,通过本课程设计,不仅加深了对图像处理理论(特别是对比度与人眼感知)的理解,更重要的是学会了如何将理论转化为可控的、参数化的算法,以及通过实验验证与对比来改进方案。

\section{展望}

未来的改进方向包括:
\begin{enumerate}
    \item \textbf{自适应参数优化}:当前参数 $(\alpha,\beta,\gamma)$ 是经验值。可利用机器学习或统计方法,根据图像的直方图分布特征自动调整参数,以适应不同内容与光照条件的图像。
    \item \textbf{边缘与纹理的分离}:进一步细化边缘与纹理的强化强度。例如,在边缘处应用更强的增强,而在均质区域适度约束,以避免过增强与噪声放大。
    \item \textbf{多尺度融合增强}:将三种窗宽下的增强结果以加权融合方式组合,而非逐尺度单独输出,可能在整体效果上更优。
    \item \textbf{与深度学习的结合}:利用卷积神经网络学习增强映射,或作为图像前处理提升后续深度学习模型的性能。
\end{enumerate}

此外,可尝试将该框架扩展到彩色图像的各通道或其他色彩空间(如 YCbCr、LAB),以及与去噪、去雾等其他增强任务的联合优化,进一步提升实用价值。

\thesisacknowledgement
感谢 MATLAB 方便的函数,老师的指导,室友免费提供的 Wi-Fi,前辈留下的 \LaTeX 模板以及先进的 AI 帮助我轻松愉快地编写 \LaTeX 文档和答疑解惑。

\thesisappendix

\chapter{源代码}

\section{Weber 对比度相关代码}
\inputminted[
    linenos,
    frame=lines,
    framesep=2mm,
    fontsize=\footnotesize,
    breaklines=true,
    numbersep=5pt
]{matlab}{./src/weber.m}

\section{Michelson 对比度相关代码}
\inputminted[
    linenos,
    frame=lines,
    framesep=2mm,
    fontsize=\footnotesize,
    breaklines=true,
    numbersep=5pt
]{matlab}{./src/michelson.m}

\section{均值平方根对比度相关代码}
\inputminted[
    linenos,
    frame=lines,
    framesep=2mm,
    fontsize=\footnotesize,
    breaklines=true,
    numbersep=5pt
]{matlab}{./src/rms.m}

\section{辅助函数代码}
\inputminted[
    linenos,
    frame=lines,
    framesep=2mm,
    fontsize=\footnotesize,
    breaklines=true,
    numbersep=5pt
]{matlab}{./src/restoreColour.m}

\thesisbibliography{reference}

\end{document}
